\documentclass{article}
\usepackage{hyperref}
\title{Specification Document}
\author{Omkar Girish Kamath}
\begin{document}
  \maketitle \tableofcontents \date
  \section{\textbf{Introduction}}
  This specification document outlines the design ,features ,functions
  and installation of the \textbf{Dynamically Linked Library of Linked
    List Functions}. The library is designed to provide commonly used
  operations for creating and manipulating linked lists. The library
  is written in C programming language and will be compatible with
  Linux operating systems. The appropriate Makefile utility files are
  written and provided as well with the library to ensure smooth and
  hiccup-less usage of the library.
  \section{\textbf{Installation in Linux Systems}}
  \subsection{Introduction to DLLs}
  {\linespread{1.0}
  Dynamically linked libraries (DLLs) are shared libraries that
  contain compiled code and data that can be linked and loaded into a
  program at run-time, rather than at compile-time. In contrast,
  static libraries are linked and loaded into the program at
  compile-time.

When a program is dynamically linked to a DLL, the program will load
the DLL into memory at runtime and resolve any unresolved external
symbols, such as function names, that are referenced in the program
code. This allows multiple programs to share the same DLL, reducing
memory usage and making updates to the library easier.  The process of
using a DLL in a program involves a few steps.  The program code
contains references to functions or data that are defined in the DLL.
The program is compiled with references to the DLL.  The program can
call functions and access data in the DLL as needed.
\subsection{Setting the Environment Variables}
The gcc dynamic linker looks for the shared object (.so) file at
runtime using \textbf{LD\_LIBRARY\_PATH} and gcc compiler looks for
header files at compilation using C\_INCLUDE\_PATH and to not mention
the path explicitly during compilation we need to add the
directory containing the shared object (.so) and the header file
(libll.h) file to the environment variable \textbf{LD\_LIBRARY\_PATH}
and C\_INCLUDE\_PATH respectively. C\_INCLUDE\_PATH is used during the
compilation process to find header files needed by the code being
compiled, while \textbf{LD\_LIBRARY\_PATH} is used at runtime to find
shared libraries needed by the executable. But to ensure that the
environment variable remembers it even after system reboots, we need
to change the configuration file of the user shell. For the sake of
example the \textbf{Bourne Again Shell (bash)} is considered.  So we
add the line \textbf{\center{export
    LD\_LIBRARY\_PATH=/path/to/sharedobjectfile/:\$LD\_LIBRARY\_PATH}}
  \textbf{\center{export
    C\_INCLUDE\_PATH=/path/to/headerfile/:\$C\_INCLUDE\_PATH}}
  }
  \section{\textbf{Intended Usage and Working of The Library}}
  \subsection{Makefile}
  
  To use this library firstly you need to use the Makefile provided
  in the directory main\_files. Open your shell terminal and use the
  command "make" in the directory main\_files. This will compile the
  library source files depending upon the dependencies and produce the
  target shared object file if its not present from before. Then you
  need to use the "make install" command which will create a directory
  in home/user/ called lib where the shared object file be copied to.
  Similarly another directory is created in home/user/ called include
  where the header file is copied to. Then the command ldconfig
  creates a symlink to the shared library in the lib directory.
  \subsection{Compilation of the File}

  Now you can use the library, the gcc command would be :
  \begin{center}
    \textbf{gcc -L /home/user/lib yourfile.c -lll}
  \end{center}
  
  \section{\textbf{Function Descriptions and their Purpose}}
  \subsection{Function Name: \_ll\_create()}
  
  Arguments: None.  \\ 
  Returns:     Head pointer to new list (ll *) \\
  Description:   Non-User function to create new lists or nodes.
  Its a private function meant not to be used by the
  user, rather used by other functions to create nodes.
  User can use the lladd() function to create and add
  nodes. \\
  
  \subsection{Function Name: lladd()}
  Arguments:     Takes in a pointer to a list (or NULL if a new list
  needs to be created) and a pointer to the data payload that will
  be added to the newly created node. \\
  Returns:       Head pointer to either a new list, or the same list
  if the head argument is non-null (ll *). \\
  Description:   It is a User function to create a new list or
  add nodes to the end of given linked list. \\
  
  \subsection{Function Name: llprint()}
  Arguments    : A starting node from to start printing from and a
  a function pointer which takes in payload data of a
  node and returns the string that is to be printed. \\
  Returns      : None \\
  Description  : User function to print all the nodes occuring after
  the starting node \\

  \subsection{ Function Name: llsearch()}
  Arguments    : A starting node pointer to begin searching from, a
  key a function pointer which takes in payload data 
  of a node and returns the string that is to be
  printed.\\
  Returns      : Pointer (ll *) containing payload data matching with
  key.\\
  Description  : User function to search for the payload data in the
  linked list\\

  \subsection{ Function Name: lldelete()}
  Arguments    : Head pointer of the list delete, node marked for
  deletion and function pointer to delete payload data
  from nodes. \\
  Returns      : NULL pointer if list has 0 elements or has 1 element
  and user data matches with that node. Original head (ll *)
  for all other cases. \\
  Description  : User function to delete node and its payload data
  which matches node marked for deletion. \\

  \subsection{ Function Name: llappend()}
  Arguments    : A node pointer 1 and a node pointer 2. \\
  Returns      : head pointer formed afte appending list with
  head node pointer 2 to tail of list formed by head
  pointer 1. \\
  Description  : User function to append linked list 2 to the end of
  linked list 1. \\

  
 
\end{document}
