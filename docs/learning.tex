\documentclass{article}
\usepackage{fancyvrb}
\title{Skills Learnt}
\author{Omkar Girish Kamath}
\begin{document}
\maketitle
\tableofcontents
\date
%%\pagebreak
\section{\textbf{Makefile}}
\subsection{Code}

\subsection{Description}
\begin{Verbatim}
  To create a shared object file using a makefile, you can use the following steps:

  Define the compiler and compiler flags to be used for creating the
  shared object file. For example, you might use gcc as the compiler
  and the -fPIC flag to generate position-independent code: makefile
  Copy code CC=gcc CFLAGS=-fPIC Define the source files and object
  files to be used in the shared object file. For example: makefile
  Copy code SRCS=foo.c bar.c OBJS=$(SRCS:.c=.o) In this example, the
  SRCS variable lists the source files for your program, and the OBJS
  variable lists the corresponding object files that will be generated.

  Define the name and version of the shared object file. For example:
  makefile

  Copy code SONAME=mylib.so VERSION=1 In this example, the SONAME
  variable specifies the name of the shared object file, and the
  VERSION variable specifies the version number.

  Define the build target for the shared object file. For example:

  Copy code $(SONAME): $(OBJS) $(CC) -shared
  -Wl,-soname,$(SONAME).$(VERSION) -o $(SONAME).$(VERSION) $(OBJS) This
  target specifies that the shared object file should be built from the
  object files generated by compiling the source files. The -shared
  flag tells the linker to generate a shared object file, and the
  -Wl,-soname option specifies the name of the shared object
  file. Finally, the -o option specifies the output file name.

  Define a clean target to remove any generated object files or shared
  object files. For example: makefile Copy code clean: rm -f $(OBJS)
  $(SONAME).$(VERSION) This target removes any object files generated
  during the build process, as well as the shared object file.

  Here is an example makefile that puts all these steps together:

  CC=gcc CFLAGS=-fPIC

  SRCS=foo.c bar.c OBJS=$(SRCS:.c=.o)

  SONAME=mylib.so VERSION=1

  $(SONAME): $(OBJS) $(CC) -shared -Wl,-soname,$(SONAME).$(VERSION)
  -o $(SONAME).$(VERSION) $(OBJS)

  clean: rm -f $(OBJS) $(SONAME).$(VERSION)

  You can run the makefile by
  navigating to the directory containing the makefile and running
  make. This will build the shared object file specified by the SONAME
  variable. You can also run make clean to remove any generated files.


  User is version number necessary for the above code ChatGPT While it
  is not strictly necessary to include a version number in the shared
  object file name, it is a common practice to do so. Including a
  version number in the shared object file name allows for multiple
  versions of the same library to be installed on a system, and for
  programs that depend on a specific version of the library to link to
  that version.

  In the makefile example I provided earlier, the version number is
  included in the shared object file name using the following line:

  javascript Copy code -Wl,-soname,$(SONAME).$(VERSION) This line uses
  the -Wl option to pass an argument to the linker, and specifies the
  soname of the shared object file as $(SONAME).$(VERSION). This means
  that the shared object file will be named mylib.so.1, where 1 is the
  version number.

  If you do not include a version number in the shared object file name,
  you may run into issues if you need to update the library or install
  multiple versions of the library on a system. Therefore, it is
  generally a good practice to include a version number in the shared
  object file name.





  User CFLAGS=-fPIC what does this do in the above code ChatGPT In the
  makefile example I provided earlier, the line CFLAGS=-fPIC sets the
  compiler flag -fPIC for all compilation steps.

  The -fPIC flag is short for "Position Independent Code", and it
  
  generates code that can be loaded and executed at any memory address.
  
  This is useful for shared libraries, as it allows the library to be
  loaded at any address in memory, rather than being tied to a specific
  address.

  When creating a shared object file, it is important to generate
  position independent code, as the
  
  shared object file may be loaded into memory at different addresses on
  different systems. By using the -fPIC flag, we ensure that the code
  generated by the compiler is position independent and can be loaded
  into memory at any address.

  Therefore, the CFLAGS=-fPIC line in the makefile is setting a compiler
  flag that is necessary for creating a shared object file.
\end{Verbatim}
\end{document}
